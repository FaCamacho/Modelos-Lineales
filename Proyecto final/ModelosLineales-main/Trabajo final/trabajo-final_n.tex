% Options for packages loaded elsewhere
\PassOptionsToPackage{unicode}{hyperref}
\PassOptionsToPackage{hyphens}{url}
%
\documentclass[
]{article}
\usepackage{amsmath,amssymb}
\usepackage{iftex}
\ifPDFTeX
  \usepackage[T1]{fontenc}
  \usepackage[utf8]{inputenc}
  \usepackage{textcomp} % provide euro and other symbols
\else % if luatex or xetex
  \usepackage{unicode-math} % this also loads fontspec
  \defaultfontfeatures{Scale=MatchLowercase}
  \defaultfontfeatures[\rmfamily]{Ligatures=TeX,Scale=1}
\fi
\usepackage{lmodern}
\ifPDFTeX\else
  % xetex/luatex font selection
\fi
% Use upquote if available, for straight quotes in verbatim environments
\IfFileExists{upquote.sty}{\usepackage{upquote}}{}
\IfFileExists{microtype.sty}{% use microtype if available
  \usepackage[]{microtype}
  \UseMicrotypeSet[protrusion]{basicmath} % disable protrusion for tt fonts
}{}
\makeatletter
\@ifundefined{KOMAClassName}{% if non-KOMA class
  \IfFileExists{parskip.sty}{%
    \usepackage{parskip}
  }{% else
    \setlength{\parindent}{0pt}
    \setlength{\parskip}{6pt plus 2pt minus 1pt}}
}{% if KOMA class
  \KOMAoptions{parskip=half}}
\makeatother
\usepackage{xcolor}
\usepackage[margin=1in]{geometry}
\usepackage{color}
\usepackage{fancyvrb}
\newcommand{\VerbBar}{|}
\newcommand{\VERB}{\Verb[commandchars=\\\{\}]}
\DefineVerbatimEnvironment{Highlighting}{Verbatim}{commandchars=\\\{\}}
% Add ',fontsize=\small' for more characters per line
\usepackage{framed}
\definecolor{shadecolor}{RGB}{248,248,248}
\newenvironment{Shaded}{\begin{snugshade}}{\end{snugshade}}
\newcommand{\AlertTok}[1]{\textcolor[rgb]{0.94,0.16,0.16}{#1}}
\newcommand{\AnnotationTok}[1]{\textcolor[rgb]{0.56,0.35,0.01}{\textbf{\textit{#1}}}}
\newcommand{\AttributeTok}[1]{\textcolor[rgb]{0.13,0.29,0.53}{#1}}
\newcommand{\BaseNTok}[1]{\textcolor[rgb]{0.00,0.00,0.81}{#1}}
\newcommand{\BuiltInTok}[1]{#1}
\newcommand{\CharTok}[1]{\textcolor[rgb]{0.31,0.60,0.02}{#1}}
\newcommand{\CommentTok}[1]{\textcolor[rgb]{0.56,0.35,0.01}{\textit{#1}}}
\newcommand{\CommentVarTok}[1]{\textcolor[rgb]{0.56,0.35,0.01}{\textbf{\textit{#1}}}}
\newcommand{\ConstantTok}[1]{\textcolor[rgb]{0.56,0.35,0.01}{#1}}
\newcommand{\ControlFlowTok}[1]{\textcolor[rgb]{0.13,0.29,0.53}{\textbf{#1}}}
\newcommand{\DataTypeTok}[1]{\textcolor[rgb]{0.13,0.29,0.53}{#1}}
\newcommand{\DecValTok}[1]{\textcolor[rgb]{0.00,0.00,0.81}{#1}}
\newcommand{\DocumentationTok}[1]{\textcolor[rgb]{0.56,0.35,0.01}{\textbf{\textit{#1}}}}
\newcommand{\ErrorTok}[1]{\textcolor[rgb]{0.64,0.00,0.00}{\textbf{#1}}}
\newcommand{\ExtensionTok}[1]{#1}
\newcommand{\FloatTok}[1]{\textcolor[rgb]{0.00,0.00,0.81}{#1}}
\newcommand{\FunctionTok}[1]{\textcolor[rgb]{0.13,0.29,0.53}{\textbf{#1}}}
\newcommand{\ImportTok}[1]{#1}
\newcommand{\InformationTok}[1]{\textcolor[rgb]{0.56,0.35,0.01}{\textbf{\textit{#1}}}}
\newcommand{\KeywordTok}[1]{\textcolor[rgb]{0.13,0.29,0.53}{\textbf{#1}}}
\newcommand{\NormalTok}[1]{#1}
\newcommand{\OperatorTok}[1]{\textcolor[rgb]{0.81,0.36,0.00}{\textbf{#1}}}
\newcommand{\OtherTok}[1]{\textcolor[rgb]{0.56,0.35,0.01}{#1}}
\newcommand{\PreprocessorTok}[1]{\textcolor[rgb]{0.56,0.35,0.01}{\textit{#1}}}
\newcommand{\RegionMarkerTok}[1]{#1}
\newcommand{\SpecialCharTok}[1]{\textcolor[rgb]{0.81,0.36,0.00}{\textbf{#1}}}
\newcommand{\SpecialStringTok}[1]{\textcolor[rgb]{0.31,0.60,0.02}{#1}}
\newcommand{\StringTok}[1]{\textcolor[rgb]{0.31,0.60,0.02}{#1}}
\newcommand{\VariableTok}[1]{\textcolor[rgb]{0.00,0.00,0.00}{#1}}
\newcommand{\VerbatimStringTok}[1]{\textcolor[rgb]{0.31,0.60,0.02}{#1}}
\newcommand{\WarningTok}[1]{\textcolor[rgb]{0.56,0.35,0.01}{\textbf{\textit{#1}}}}
\usepackage{longtable,booktabs,array}
\usepackage{calc} % for calculating minipage widths
% Correct order of tables after \paragraph or \subparagraph
\usepackage{etoolbox}
\makeatletter
\patchcmd\longtable{\par}{\if@noskipsec\mbox{}\fi\par}{}{}
\makeatother
% Allow footnotes in longtable head/foot
\IfFileExists{footnotehyper.sty}{\usepackage{footnotehyper}}{\usepackage{footnote}}
\makesavenoteenv{longtable}
\usepackage{graphicx}
\makeatletter
\def\maxwidth{\ifdim\Gin@nat@width>\linewidth\linewidth\else\Gin@nat@width\fi}
\def\maxheight{\ifdim\Gin@nat@height>\textheight\textheight\else\Gin@nat@height\fi}
\makeatother
% Scale images if necessary, so that they will not overflow the page
% margins by default, and it is still possible to overwrite the defaults
% using explicit options in \includegraphics[width, height, ...]{}
\setkeys{Gin}{width=\maxwidth,height=\maxheight,keepaspectratio}
% Set default figure placement to htbp
\makeatletter
\def\fps@figure{htbp}
\makeatother
\setlength{\emergencystretch}{3em} % prevent overfull lines
\providecommand{\tightlist}{%
  \setlength{\itemsep}{0pt}\setlength{\parskip}{0pt}}
\setcounter{secnumdepth}{-\maxdimen} % remove section numbering
\usepackage{booktabs}
\usepackage{longtable}
\usepackage{array}
\usepackage{multirow}
\usepackage{wrapfig}
\usepackage{float}
\usepackage{colortbl}
\usepackage{pdflscape}
\usepackage{tabu}
\usepackage{threeparttable}
\usepackage{threeparttablex}
\usepackage[normalem]{ulem}
\usepackage{makecell}
\usepackage{xcolor}
\ifLuaTeX
  \usepackage{selnolig}  % disable illegal ligatures
\fi
\usepackage{bookmark}
\IfFileExists{xurl.sty}{\usepackage{xurl}}{} % add URL line breaks if available
\urlstyle{same}
\hypersetup{
  pdftitle={Entrega final - Determinación del peso de los peces},
  pdfauthor={Fabricio Camacho,Matias Bajac},
  hidelinks,
  pdfcreator={LaTeX via pandoc}}

\title{Entrega final - Determinación del peso de los peces}
\author{Fabricio Camacho,Matias Bajac}
\date{2024-06-25}

\begin{document}
\maketitle

\section{Introducción}\label{introducciuxf3n}

A lo largo de este trabajo vamos a interpretar y modelar datos
referentes a las dimensiones de peces de la costa de Finlandia. Las
variables con las que contamos son su peso, longitud, ancho y altura,
además de la especie de cada pez.

El objetivo final es poder predecir el peso de los peces mediante las
restantes variables, estas son:

\begin{longtable}[]{@{}cc@{}}
\toprule\noalign{}
Variable & Descripción \\
\midrule\noalign{}
\endhead
\bottomrule\noalign{}
\endlastfoot
Especie & Bream, Parki, Perch, Pike, Roach, Smelt, Whitefish \\
Peso\_gr & Peso del pez en gramos \\
Altura\_cm & Altura en centímetros \\
Ancho\_cm & Ancho en centímetros \\
Longitud1 & Desde la nariz al comienzo de la cola \\
Longitud2 & Desde la punta de la nariz hasta la muesca de la cola \\
Longitud3 & Desde la nariz al final de la cola \\
\end{longtable}

La base de datos cuenta con 159 peces donde uno de ellos será quitado de
la misma por tenér un 0 en la varialbe de peso, consiguiendo finalmente
un total de 158 peces.

A efectos de tener un primer acercamiento con la estructura de los
datos, se obtienen algunas estadisticas descripitvas, como la
correlacion entre las variables cuantitativas y un diagrama de caja para
visualizar el peso en relacion a cada especie

\section{Metodologia}\label{metodologia}

La idea es implementar las tecnicas de analisis estudiadas en el curso
de Modelos Lineales, en partiuclar, el modelo de regresion multiple.

En una primera instancia, se procede a hacer un analisis explotatorio de
los datos. Luego pasamos a una primera etapa de diagnostico, dado que la
intención es poder inferir en una generalidad de peces, hay ciertos
supuestos que tenemos que validar , estos son:

\begin{itemize}
\item
  Multicolinealidad: Donde nos va a interesar que ninguna variable sea
  combinación lineal del resto.
\item
  Homoscedasticidad: Donde la varianza de los residuos para cada pez son
  iguales.
\item
  Normalidad: Donde los residuos de los estimados tienen una
  distribución normal.
\end{itemize}

Para estudiar el supuesto de la multicolinealidad aproximada, en el cual
nos sirve para quedarnos con las variables explicativas siguiendo el
criterio de Vif\textless5. Para la Homoscedasticidad vamos a aplicar el
test de Bresuch-Pagan, y para la normalidad el test de
Kolmogorov-Smirnov.

Este analísis diagnóstico es aplicado para cada modelo candidato a
responder nuestras inquietudes de investigación a efectos de encontrar
el mejor, o en otras palabras el que pueda explicar en mayor medida la
varianza.

En una siguiente etapa, se hizo un analisis ANOVA y ANCOVA. Bajo la hi
potesis de que hay modelos mejores que otros y variables que puedan
explicar de mejor forma el peso de los peces, es que tendremos
particular interés en ver como interactuan distintas variables,
haciendolas complementarse entre sí.

Para finalizar, se usaron técnicas de cross validation para evalular
todos los modelos.

A efectos de tener un primer acercamiento con la estructura de los
datos, se obtienen algunas estadisticas descripitvas, como la
correlacion entre las variables cuantitativas y un diagrama de caja para
visualizar el peso en relación a cada especie

\section{Resultados}\label{resultados}

\subsection{Análisis exploratorio de los
datos}\label{anuxe1lisis-exploratorio-de-los-datos}

\begingroup\fontsize{8}{10}\selectfont

\begin{longtable}[t]{llllllll}
\toprule
 & Especie & Peso\_gr & Longitud1 & Longitud2 & Longitud3 & Altura\_cm & Ancho\_cm\\
\midrule
 & Bream    :35 & Min.   :   5.9 & Min.   : 7.50 & Min.   : 8.40 & Min.   : 8.80 & Min.   :14.50 & Min.   : 8.70\\
 & Parkki   :11 & 1st Qu.: 121.2 & 1st Qu.:19.15 & 1st Qu.:21.00 & 1st Qu.:23.20 & 1st Qu.:24.23 & 1st Qu.:13.40\\
 & Perch    :56 & Median : 281.5 & Median :25.30 & Median :27.40 & Median :29.70 & Median :27.00 & Median :14.60\\
 & Pike     :17 & Mean   : 400.8 & Mean   :26.29 & Mean   :28.47 & Mean   :31.28 & Mean   :28.31 & Mean   :14.11\\
 & Roach    :19 & 3rd Qu.: 650.0 & 3rd Qu.:32.70 & 3rd Qu.:35.75 & 3rd Qu.:39.67 & 3rd Qu.:37.70 & 3rd Qu.:15.30\\
\addlinespace
 & Smelt    :14 & Max.   :1650.0 & Max.   :59.00 & Max.   :63.40 & Max.   :68.00 & Max.   :44.50 & Max.   :20.90\\
 & Whitefish: 6 & NA & NA & NA & NA & NA & NA\\
\bottomrule
\end{longtable}
\endgroup{}

\begin{figure}
\centering
\includegraphics{trabajo-final_n_files/figure-latex/unnamed-chunk-2-1.pdf}
\caption{matriz de correlacion entre las variables cuantitativas.}
\end{figure}

Podemos observar mediante la matriz de correlacion que existe
correlacion fuerte entre las variables Longitud de los peces, lo que
podria casuar problemas de multicolinealidad aproximada, como tambien
problemas en la homoscedasticidad de la varianza de los residuos. Parece
razonable a priori incluir solo una variable respecto a la longitud del
pez para predecir su peso.

\begin{figure}
\centering
\includegraphics{trabajo-final_n_files/figure-latex/unnamed-chunk-3-1.pdf}
\caption{Diagrama de caja, relación entre cada especie y el peso}
\end{figure}

\section{Análisis gráfico}\label{anuxe1lisis-gruxe1fico}

\begin{figure}
\centering
\includegraphics{trabajo-final_n_files/figure-latex/unnamed-chunk-4-1.pdf}
\caption{Diagrama de dispresion de los regresores respecto al peso}
\end{figure}

\subsubsection{Análisis de supuestos sobre modelos
lineales}\label{anuxe1lisis-de-supuestos-sobre-modelos-lineales}

\paragraph{Multicolinealidad}\label{multicolinealidad}

\begin{verbatim}
##   Longitud1   Longitud2   Longitud3   Altura_cm    Ancho_cm 
## 1817.971125 2391.862631  315.886137    4.724397    2.253745
\end{verbatim}

\begin{verbatim}
##  Longitud1  Longitud3  Altura_cm   Ancho_cm 
## 229.258810 233.347914   4.667798   1.812460
\end{verbatim}

\begin{verbatim}
## Longitud1 Altura_cm  Ancho_cm 
##  1.001651  1.258646  1.258136
\end{verbatim}

En esta instancia analizamos la multicolinealidad, la idea es ver si hay
variables que sean combinación lineal de otras, o sea, que en tengan la
misma información. En caso de que el \emph{vif} sea alto (mayor a 5),
quitaremos la variable con el \emph{vif} más alto. La presencia de
multicolinealidad impide sobretodo la posibilidad de analizar el efecto
de una variable predictora sobre lo que queremos predecir, en nuestro
caso el peso.

En el primer paso de este análisis hallamos \emph{vif} elevados en las
distintas longitudes, aquí volvemos confirmar lo estudiado en el
análisis de correlación previo, donde las longitudes están altamente
correlacionadas, lo cual indica que contar todas las medidas es
inviable.

\[VIF_{j} = \frac{1}{1-R^{2}_{j}}\]

De esta forma las variables finales seran \emph{Longitud1 , Altura\_cm,
Ancho\_cm}

\subsubsection{Modelos}\label{modelos}

\paragraph{Modelo 1}\label{modelo-1}

Como ya se menciono,el primer modelo estimado consiste en la regresión
de la variable peso con las variables explicativas que fueron
seleccionadas en el paso de multicolinealidad. El modelo queda
esepecificado como:

\[peso_{i}=  \beta_0 +\ \beta_1Longitud1_{i} + \beta_2Altura_{i}\ +\beta_3Ancho_{i}\ +\ \epsilon_{i}\]

\subparagraph{Diagnostico del modelo}\label{diagnostico-del-modelo}

Homoscedasticidad

Aquí se opto por recurrir a un análisis visual de los residuos
externamente estudientizados del modelo. A continuacion vemos el grafico
de los residuos en el eje de las ordenadas. Con un \(\alpha=0.05\)
rechazamos la hipotesis nula, por lo que podemos afirmar que no hay
homoscedasticidad con un \(p-valor < 0.0001\).

\[H_0)\ E(\epsilon_{i}^{2}) =  \sigma^2\ vs\ H_1)\ no\ H_0\]

\begin{figure}
\centering
\includegraphics{trabajo-final_n_files/figure-latex/unnamed-chunk-7-1.pdf}
\caption{Análisis de los residuos externamente studientizados del modelo
1}
\end{figure}

\begin{verbatim}
## # A tibble: 1 x 5
##   statistic  p.value parameter method                alternative
##       <dbl>    <dbl>     <dbl> <chr>                 <chr>      
## 1      81.0 1.87e-17         3 Koenker (studentised) greater
\end{verbatim}

\subsection{Normalidad}\label{normalidad}

El histograma de los residuos externamente estudentizados no se parece a
una distribucion normal en los residuos. Ademas, el test de normalidad
de Kolmogorov-Smirnov, segun el criterio del p\_valor y para un
\(\alpha=0.5\) se rechaza la hipotesis nula de normalidad de los
residuos.

El modelo queda descartado al no superar el supuesto de
homoscedasticidad.

\begin{verbatim}
## 
##  Jarque Bera Test
## 
## data:  datos$t_i
## X-squared = 3.7807, df = 2, p-value = 0.151
\end{verbatim}

\begin{verbatim}
## 
##  Asymptotic one-sample Kolmogorov-Smirnov test
## 
## data:  datos$t_i
## D = 0.13593, p-value = 0.005824
## alternative hypothesis: two-sided
\end{verbatim}

\begin{figure}
\centering
\includegraphics{trabajo-final_n_files/figure-latex/unnamed-chunk-8-1.pdf}
\caption{Histograma de los Residuos studientizados}
\end{figure}

\paragraph{Modelo 2}\label{modelo-2}

Como segundo modelo se estimo una regresion con transformacion
logaritmica tanto en la variable dependiente como en la variables
explicativas

\[Log(Peso_i) = \beta_0\ + \beta_1Log(Longitud_1)\ + \beta_2Log(Altura_i)\  +  \beta_3Log(Ancho_{i}) +\epsilon_i \]

\subsection{Diagnostico del modelo}\label{diagnostico-del-modelo-1}

\subsection{Homoscedasticidad}\label{homoscedasticidad-1}

\includegraphics{trabajo-final_n_files/figure-latex/unnamed-chunk-10-1.pdf}

Normalidad

El histograma de los residuos estandarizados se parecerse a una
distribucion normal en los residuos. Ademas, el test de normalidad de
Kolmogorov-Smirnov , segun el criterio del p\_valor y para un
\(\alpha=0.5\) no rechaza la hipotesis nula de normalidad de los
residuos.

\includegraphics{trabajo-final_n_files/figure-latex/unnamed-chunk-11-1.pdf}

\begin{verbatim}
## 
##  Asymptotic one-sample Kolmogorov-Smirnov test
## 
## data:  datos$t_i
## D = 0.059245, p-value = 0.6361
## alternative hypothesis: two-sided
\end{verbatim}

\includegraphics{trabajo-final_n_files/figure-latex/unnamed-chunk-11-2.pdf}

\paragraph{Modelo 3}\label{modelo-3}

El modelo 3 cumple con todos los supuestos y es óptimo para realizar el
análisis de inferencia y responder las preguntas de investigación. De
todas formas, podemos llegar a la conclusión de que el aporte de las
variables \emph{Ancho\_cm} y \emph{Altura\_cm} es marginal, conecntrando
en \emph{Longitud1} la mayor explicación de la varianza de los pesos.
Siendo este un modelo mas parsimonioso para poder explicar el peso de
los peses.

De esta manera, de aquí en más vamos a trabajar con el Modelo 3.

\[Log(Peso_{i}) = \beta_0\ + \beta_1Log(Longitud1_{i})\ \ +\ \epsilon_i\]

\begin{Shaded}
\begin{Highlighting}[]
\NormalTok{mod3 }\OtherTok{=} \FunctionTok{lm}\NormalTok{(}\FunctionTok{log}\NormalTok{(Peso\_gr) }\SpecialCharTok{\textasciitilde{}} \FunctionTok{log}\NormalTok{(Longitud1)  , }\AttributeTok{data =}\NormalTok{ datos)}



\FunctionTok{summary}\NormalTok{(mod3)}
\end{Highlighting}
\end{Shaded}

\begin{verbatim}
## 
## Call:
## lm(formula = log(Peso_gr) ~ log(Longitud1), data = datos)
## 
## Residuals:
##      Min       1Q   Median       3Q      Max 
## -0.90870 -0.07280  0.07773  0.26639  0.50636 
## 
## Coefficients:
##                Estimate Std. Error t value Pr(>|t|)    
## (Intercept)    -4.62769    0.23481  -19.71   <2e-16 ***
## log(Longitud1)  3.14394    0.07296   43.09   <2e-16 ***
## ---
## Signif. codes:  0 '***' 0.001 '**' 0.01 '*' 0.05 '.' 0.1 ' ' 1
## 
## s: 0.3704 on 156 degrees of freedom
## Multiple R-squared: 0.9225,
## Adjusted R-squared: 0.922 
## F-statistic:  1857 on 1 and 156 DF,  p-value: < 2.2e-16
\end{verbatim}

\subsection{Significacion individual}\label{significacion-individual}

Para cada uno de las variables explicativas se realiza la siguiente
prueba de hipotesis:

\[H_0) B_{i} = 0\ vs\ H_1) B_{i} \neq 0\] Con región critica
\(RC = \Big\{ \Big({X}{y}\Big) \, \Big/ \, |t| \geq t_{n-k-1} (1 - \, ^\alpha\!/_2) \Big\}\)

Se usa el estadístico:
\(t=\frac{\hat{\beta_i}}{\hat{V}\hat{(\beta_i)}}  \sim t_{n-k-1}\)\$

Siguiendo el criterio del p\_valor, la evidencia empirica sugiere que
las variables Longitud1 en centimetros es individualmente significativa
para explicar el peos del pez a un nivel de confianza del 5\%.

\subsection{Signficacion global del
modelo}\label{signficacion-global-del-modelo}

Siguiendo el criterio del p\_valor, a un nivel del 5\%, la evidencia
empirica sugiere que el modelo es globalmente significativo. Esto
implica que, dada la evidencia empirica con la que se cuenta, no es
posible rechazar la hipotesis de que las variables explicativas usadas
no contribuyen a explicar el peso del pez.

\#\#Homoscedasticidad

\begin{verbatim}
## # A tibble: 1 x 5
##   statistic p.value parameter method                alternative
##       <dbl>   <dbl>     <dbl> <chr>                 <chr>      
## 1     0.885   0.347         1 Koenker (studentised) greater
\end{verbatim}

\includegraphics{trabajo-final_n_files/figure-latex/fig-1.pdf}

\subsection{Normalidad}\label{normalidad-2}

\includegraphics{trabajo-final_n_files/figure-latex/unnamed-chunk-14-1.pdf}

\begin{verbatim}
## 
##  Asymptotic one-sample Kolmogorov-Smirnov test
## 
## data:  datos$t_i
## D = 0.18507, p-value = 3.984e-05
## alternative hypothesis: two-sided
\end{verbatim}

\includegraphics{trabajo-final_n_files/figure-latex/unnamed-chunk-14-2.pdf}

\subsection{ANOVA a 1 vía}\label{anova-a-1-vuxeda}

El objetivo es estudiar si existe igualdad de medias entre las
categorias de la variable Especie. Haciendo el analisis de varianza a
una via, nos plantemos 2 modelos, uno solo con la constante y otro
especificando la especie.

El modelo con la constante queda especificado de la siguiente manera:

\[Peso_{ij} = \mu + \epsilon_{ij}\]

mieintras que si le agregamos el efecto especie queda:

\[Peso_{ij} = \mu + Especie_{ij} +  \epsilon_{ij}\]

A un nivel de significación del 5\%, podemos afirmar que tenemos
evidencia suficiente para rechaza la hipotesis nula de igualdad de
medias.

\begingroup\fontsize{8}{10}\selectfont

\begin{longtable}[t]{lrrrr}
\toprule
Especie & media(peso) & desvio(peso) & min(peso) & max(peso)\\
\midrule
Bream & 617.83 & 209.21 & 242.0 & 1000.0\\
Parkki & 154.82 & 78.76 & 55.0 & 300.0\\
Perch & 382.24 & 347.62 & 5.9 & 1100.0\\
Pike & 718.71 & 494.14 & 200.0 & 1650.0\\
Roach & 160.05 & 83.53 & 40.0 & 390.0\\
\addlinespace
Smelt & 11.18 & 4.13 & 6.7 & 19.9\\
Whitefish & 531.00 & 309.60 & 270.0 & 1000.0\\
\bottomrule
\end{longtable}
\endgroup{}

\begin{verbatim}
## Analysis of Variance Table
## 
## Response: log(Peso_gr)
##            Df  Sum Sq Mean Sq F value    Pr(>F)    
## Especie     6 187.941  31.323   53.64 < 2.2e-16 ***
## Residuals 151  88.177   0.584                      
## ---
## Signif. codes:  0 '***' 0.001 '**' 0.01 '*' 0.05 '.' 0.1 ' ' 1
\end{verbatim}

\includegraphics{trabajo-final_n_files/figure-latex/unnamed-chunk-15-1.pdf}
\includegraphics{trabajo-final_n_files/figure-latex/unnamed-chunk-15-2.pdf}

\includegraphics{trabajo-final_n_files/figure-latex/unnamed-chunk-16-1.pdf}
\includegraphics{trabajo-final_n_files/figure-latex/unnamed-chunk-16-2.pdf}

\begin{verbatim}
## 
##  Asymptotic one-sample Kolmogorov-Smirnov test
## 
## data:  datos$t_i
## D = 0.093715, p-value = 0.1246
## alternative hypothesis: two-sided
\end{verbatim}

\includegraphics{trabajo-final_n_files/figure-latex/unnamed-chunk-16-3.pdf}

\subsection{Ancova}\label{ancova}

Sabiendo que la variable especie es significativo para predecir el peso
de cada pez, vamos a plantearnos un modelo en el cual tenga una variable
cuantitativa (longitud1) y la variable categorica Especie.

\[Peso_i = \beta_0\ +\ \beta_1log(Longitud_i)\ +\beta_2Especie_i\  +\ \epsilon_i\]

Luego, para ver si existe un efecto Especie sobre la pendiente, debemos
plantearnos un modelo con interaccion. El modelo con interaccion queda
definido como:

\[Peso_i = \beta_0\ +\ \beta_1log(Longitud_i)\ +\beta_2Especie_i\ +\ \beta_3log(Longitud1_i):Especie_i\ +\ \epsilon_i\]

Ahora pasamos a estudiar si efectivamente existe igualdad de pendientes
entre las especies. No hay evidencia suficiente para decir que la
pendiente sean distinta entre las especies.

\[H_0) B_{i} = 0\ vs\ H_1) B_{i} \neq 0\]

Se puede observar tambien mediante un grafico de puntos que existe una
relacion lineal entre el peso y la longitud. Mas aun haciendo una
transformacion logaritmica a ambas variables.

Al plantearnos el modelo con la covariable longitud1 y especie , podemos
ver que es suficiente para predecir el peso. Por lo que existe un efecto
de la longitud y tambien un efecto de la especie.

\begin{verbatim}
## Analysis of Variance Table
## 
## Model 1: log(Peso_gr) ~ log(Longitud1) + Especie
## Model 2: log(Peso_gr) ~ log(Longitud1) + Especie + log(Longitud1):Especie
##   Res.Df    RSS Df Sum of Sq      F Pr(>F)
## 1    150 2.0118                           
## 2    144 1.9830  6  0.028848 0.3491 0.9095
\end{verbatim}

\includegraphics{trabajo-final_n_files/figure-latex/unnamed-chunk-17-1.pdf}

No rechazamos \(H_0)\) por lo tanto vemos que las pendientes son iguales
entre las especies con un nivel de significación del 5\%.

\begin{verbatim}
## [1] 0.9157195
\end{verbatim}

\includegraphics{trabajo-final_n_files/figure-latex/unnamed-chunk-18-1.pdf}

\begin{verbatim}
## # A tibble: 1 x 5
##   statistic p.value parameter method                alternative
##       <dbl>   <dbl>     <dbl> <chr>                 <chr>      
## 1      2.60   0.919         7 Koenker (studentised) greater
\end{verbatim}

\includegraphics{trabajo-final_n_files/figure-latex/unnamed-chunk-18-2.pdf}
\includegraphics{trabajo-final_n_files/figure-latex/unnamed-chunk-18-3.pdf}

\begin{verbatim}
## 
##  Asymptotic one-sample Kolmogorov-Smirnov test
## 
## data:  datos$t_i
## D = 0.05098, p-value = 0.8061
## alternative hypothesis: two-sided
\end{verbatim}

\includegraphics{trabajo-final_n_files/figure-latex/unnamed-chunk-18-4.pdf}

\subsection{Cross-Validation}\label{cross-validation}

El objetivo de la validacion cruzada es separar primero la base en 2,
una de testeo y otra de entrenamiento. Luego se usa esta ultima para
estimar el modelo. Luego se utiliza el modelo para predecir la variable
peso en la base de testeo. Por ultimo se calcula el ECM de dichas
predicciones

\[CV_{[n]}= \frac{1}{k}\sum_{i=1}^kECM_{i}\]

\begingroup\fontsize{8}{10}\selectfont

\begin{longtable}[t]{lr}
\toprule
modelo & k\_folds\_cv\\
\midrule
modelo 1 & 1.76\\
modelo 2 & 0.60\\
modelo 3 & 0.01\\
\bottomrule
\end{longtable}
\endgroup{}

\section{Conclusiones}\label{conclusiones}

Mediante ensayo y error llegamos a la conclusion de que el modelo con
una variable explicativa alcanza para poder predecir el peso de los
peces, lo cual lo hace un modelo mas eficiente que la otra alternativa,
por mas que el indicador del \(R^2\) ajustado sea mas chico que otro
modelo con mas covariables, no es determinante para no quedarnos con
este modelo.

Como era de esperarse, al hacer el analisis de varianza a una via,
podemos afirmar que existe diferencia de medias entre los grupos, siendo
especie una buena variable para distinguir el peso de los peces.

En tanto al analisis de covarianza (ANCOVA), vimos que las covariables
longitud y especie, son buenas predictoras para predecir el peso, y no
importa la interaccion entre ellas.

\end{document}
